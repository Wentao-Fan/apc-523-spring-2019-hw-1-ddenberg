\documentclass[12pt]{article}
\usepackage{amsmath}
\usepackage[margin=1in]{geometry}

\usepackage{amssymb}

\title{Homework 1: Problem 1}
\author{David Denberg}
\date{\today}

\begin{document}
\maketitle

\noindent
Let 
\[
x = \pm (b_0.b_1 b_2...) 2^e
\]
and 
\[
\mathrm{rd}(x) = \pm (b_0.b_1 b_2...b_{p-1}) 2^e
\]
where $b_{p-1} = 1$ if $b_p = 1$. We take the two cases separately. 

\noindent
Case 1 ($b_p = 0$):
\begin{align*}
\frac{|x - \mathrm{rd}(x)|}{|x|} &= \frac{|(b_0.b_1 b_2 ...)_2 \times 2^e - (b_0.b_1 b_2...b_{p-1})_2 \times 2^e|}{|(b_0.b_1 b_2 ...)_2 \times 2^e|} \\
&= \frac{|(0.b_{p+1} b_{p+2} ...)_2 \times 2^{e-p}|}{|(b_0.b_1 b_2 ...)_2 \times 2^e|} \\
&= \frac{(0.b_{p+1} b_{p+2} ...)_2}{(b_0.b_1 b_2 ...)_2} \times 2^{-p}
\end{align*}
To maximize this quantity, the denominator must be minimized and the numerator, maximized. $b_0$ in the denominator must be 1 as the quantity is normalized so the minimum value in the denominator is 1. If we choose $b_i = 1$ for $i = p+1,p+2,...$ then the numerator must equal:
\[
\sum_{i=1}^{\infty} 2^{-i} = 1
\]
as it is a geometric series. Then the maximum absolute relative error when $b_p = 0$ is $2^{-p}$.

\noindent
Case 2 ($b_p = 1$):
\begin{align*}
\frac{|x - \mathrm{rd}(x)|}{|x|} &= \frac{|(b_0.b_1 b_2 ...)_2 \times 2^e - (b_0.b_1 b_2...b_{p-2} 1)_2 \times 2^e|}{|(b_0.b_1 b_2 ...)_2 \times 2^e|} \\
&= \frac{|((b_{p-1} - 1).b_{p} b_{p+1} ...)_2 \times 2^{e-(p-1)}|}{|(b_0.b_1 b_2 ...)_2 \times 2^e|} \\
&= \frac{|((b_{p-1} - 1).b_{p} b_{p+1} ...)_2|}{(b_0.b_1 b_2 ...)_2} \times 2^{-(p-1)}
\end{align*}
To maximize this quantity let $b_{p-1} = 0$ and choose $b_i = 1$ for $i = p,p+1,...$. The numerator must then equal:
\[
\sum_{i=0}^{\infty} 2^{-i} = 2
\]
The denominator is the same as in Case 1, so the maximum absolute relative error when $b_p = 1$ is $2^{-p}$.

\end{document}