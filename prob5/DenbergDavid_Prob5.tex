\documentclass[12pt]{article}
\usepackage{amsmath}
\usepackage[margin=1in]{geometry}
\usepackage[makeroom]{cancel}

\usepackage{amssymb}
\usepackage{graphicx}

\title{Homework 1: Problem 5}
\author{David Denberg}
\date{\today}

\begin{document}
\maketitle
The value of $n_{\mathrm{stop}}$ was $10^{13}$, and the final value was 2.71611003409. We can write the floating point error for the function as follows:
\begin{align*}
\mathrm{fl}(v^n) &= v^n(1 + n \epsilon_v) \quad	|\epsilon_v| < \mathrm{eps}
\end{align*}
for some value v (note that the errors from computing v are neglected if n is very large). As n grows to $10^{12}$ or $10^{13}$ then the quantity $n \cdot$ eps becomes noticeably large. When $n = 10^{13}$ and eps = $2^{-53}$ then $n \cdot$ eps $= 0.0011$ which is substantial. At $n = 10^{13}$ it is the 4th significant digit that differs from the previous value in the sequence.

\begin{align*}
n &= 10^0, s = 2.00000000000
\\
n &= 10^1, s = 2.59374246010
\\
n &= 10^2, s = 2.70481382942
\\
n &= 10^3, s = 2.71692393224
\\
n &= 10^4, s = 2.71814592682
\\
n &= 10^5, s = 2.71826823719
\\
n &= 10^6, s = 2.71828046910
\\
n &= 10^7, s = 2.71828169413
\\
n &= 10^8, s = 2.71828179835
\\
n &= 10^9, s = 2.71828205201
\\
n &= 10^{10}, s = 2.71828205323
\\
n &= 10^{11}, s = 2.71828205336
\\
n &= 10^{12}, s = 2.71852349604
\\
n &= 10^{13}, s = 2.71611003409
\end{align*}



\end{document}