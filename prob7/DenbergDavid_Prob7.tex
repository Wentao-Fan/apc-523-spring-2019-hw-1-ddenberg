\documentclass[12pt]{article}
\usepackage{amsmath}
\usepackage[margin=1in]{geometry}
\usepackage[makeroom]{cancel}

\usepackage{amssymb}
\usepackage{graphicx}

\title{Homework 1: Problem 7}
\author{David Denberg}
\date{\today}

\begin{document}
\maketitle
\noindent
\textbf{Part A}

Wilkinson's polynomial, with $n = 20$ is: 
\begin{align*}
w(x) =\; &x^{20} - 210 x^{19} + 20615 x^{18} - 1256850 x^{17} + 53327946 x^{16} - 1672280820 x^{15} + 
\\
&40171771630 x^{14} - 756111184500 x^{13} + 11310276995381 x^{12} - 135585182899530 x^{11} + 
\\
&1307535010540395 x^{10} - 10142299865511450 x^9 + 63030812099294896 x^8 - 
\\
&311333643161390640 x^7 + 1206647803780373360 x^6 - 3599979517947607200 x^5 +
\\
&8037811822645051776 x^4 - 12870931245150988800 x^3 + 13803759753640704000 x^2 - 
\\
&8752948036761600000 x + 2432902008176640000
\end{align*}

\noindent
\textbf{Part B}

Using MATLAB's `fzero' function with an initial guess of $x = 21$, the closest root it found was: 20.000001304... . Using AMTALB's `roots' function to find all roots the largest root it found was: 20.000244798... . `fzero' gave a better estimate of the root than `roots', however it was still off by $1.304 \cdot 10^{-6}$.

\noindent
\textbf{Part C}

The largest root changes with delta as follows:

\begin{table}[h]
\centering
\begin{tabular}{|l|l|}
\hline
$\delta$      & Root   \\ \hline
$10^{-8}$ & $20.6476 + 1.1869i$ \\ \hline
$10^{-4}$ & $23.1490 + 2.7410i$ \\ \hline
$10^{-4}$ & $28.4002 + 6.5104i$ \\ \hline
$10^{-2}$ & $38.4782 + 20.834i$ \\ \hline
\end{tabular}
\end{table}

\noindent
\textbf{Part D}

The roots 16 and 17 change from 16.0411 and 16.9743 to $16.7308 + 2.8127i$ and $16.7308 - 2.8127i$ respectively.

\noindent
\textbf{Part E}

(i)
\[
p(x) = a_0 + ... + a_i x^i + ... + a_{n-1} x^{n+1} + x^n
\]
Let a perturbation in $p(x)$ from a perturbation in $a_i$ be defined as:
\[
\bar{p}(x) = a_0 + ... + (a_i + \delta a_i) x^i + ... + a_{n-1} x^{n+1} + x^n
\]
Then from a perturbation in $x$:
\begin{align*}
\bar{p}(x + \delta x) &= a_0 + ... + (a_i + \delta a_i) (x + \delta x)^i + ... + a_{n-1} (x + \delta x)^{n+1} + (x + \delta x)^n
\\
&= p(x + \delta x) + \delta  a_i (x + \delta x)^i
\end{align*}
After subtracting and dividing by $\delta x$ we get:
\begin{align*}
\bigg|\frac{p(x + \delta x) - p(x)}{\delta x}\bigg| &= \bigg| \frac{a_i (x + \delta x)^i}{\delta x} \bigg|
\\
\implies |p'(x)| &= \bigg| \frac{a_i x^i}{\delta x} \bigg|
\end{align*}
We want to find the condition number $\kappa_i$ in:
\[
\bigg|\frac{\delta x}{x} \bigg| = \kappa_i \bigg| \frac{\delta a_i}{a_i} \bigg|
\]
After rearranging we get:
\[
\kappa_i = \frac{|a_i x^{i-1}|}{|p'(x)|}
\]
Thus the condition number at the root $\Omega_k$ is:
\[
(\mathrm{cond} \; \Omega_k)(a_i) = \frac{|a_i \Omega_k^{i-1}|}{|p'(\Omega_k)|}
\]
And $(\mathrm{cond} \; \Omega_k)(\vec{a})$ is:
\[
(\mathrm{cond} \; \Omega_k)(\vec{a}) = \sum_{i=0}^{n-1} \frac{|a_i \Omega_k^{i-1}|}{|p'(\Omega_k)|}
\]

(ii) The condition numbers for $\Omega_k =$ 14, 16, 17, and 20 are: 5.401185e+13, 3.544043e+13, 1.812241e+13, and 1.378468e+11 respectively.

(iii) Because the condition numbers for the `true' function are so large an algorithm cannot be any better. There would not be any improvement by trying to come up with a clever algorithm without introducing some arbitrarily precise floating point representation of a number.


\end{document}